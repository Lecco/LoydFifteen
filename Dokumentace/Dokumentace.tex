\documentclass[12pt]{report}
\usepackage[czech]{babel}
\usepackage[utf8]{inputenc}
\usepackage{a4wide}
\pagestyle{headings}
\author{Oldřich Pulkrt}
\title{Dokumentace k semestrální práci z předmětu KIV/PC}

\begin{document}
\maketitle
\tableofcontents

\chapter{Zadání}
Naprogramujte v ANSI C přenositelnou1 konzolovou aplikaci, která jako vstup načte z parametru na příkazové řádce výchozí stav hlavolamu \uv{Loydova patnáctka} a tento hlavolam vyřeší, tj. převede jej posloupností povolených tahů do základní pozice (posloupnost od jedné do nejvyššího čísla zadání a na poslední pozici - vpravo dole - je prázdné pole).
\par
Celé zadání lze nalézt na adrese http://www.kiv.zcu.cz/studies/predmety/pc/doc/work/sw2012-02.pdf

\chapter{Analýza úlohy}

\section{Řešitelnost zadání}

\section{Datové struktury}

\section{Algoritmy}

\section{Výběr vhodných datových struktur a algoritmů}

\chapter{Popis implementace}

\chapter{Uživatelská příručka}

\chapter{Závěr}

\end{document}