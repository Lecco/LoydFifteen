\documentclass[12pt,titlepage]{article}
\usepackage[czech]{babel}
\usepackage[utf8]{inputenc}
\usepackage{a4wide}
\usepackage{hyperref}
\pagestyle{headings}
\author{Oldřich Pulkrt}
\title{Dokumentace k semestrální práci z předmětu KIV/PC}

\begin{document}
\maketitle
\tableofcontents
\newpage

\section{Zadání}
\par
Naprogramujte v ANSI C přenositelnou1 konzolovou aplikaci, která jako vstup načte z parametru na příkazové řádce výchozí stav hlavolamu \uv{Loydova patnáctka} a tento hlavolam vyřeší, tj. převede jej posloupností povolených tahů do základní pozice (posloupnost od jedné do nejvyššího čísla zadání a na poslední pozici - vpravo dole - je prázdné pole).
\newline
\par
Celé zadání lze nalézt na adrese \url{http://www.kiv.zcu.cz/studies/predmety/pc/doc/work/sw2012-02.pdf}

\section{Analýza úlohy}
V této sekci se budu postupně zabývat dvěma hlavními problémy u této úlohy. Prvním z nich je ukázat, zda je zadání opravdu řešitelné, a druhým problémem bude samotné vyřešení. Předem předpokládáme, že uživatel nemusí znát správný tvar zadání, takže se na vstupu může objevit cokoli a program sám se musí ujistit o tom, že je zadání v pořádku. Bude-li cokoli v nepořádku, program na to upozorní, viz uživatelská příručka.

\subsection{Řešitelnost zadání}
Nejprve musíme zkontrolovat, zda je zadání ve správném tvaru. Musíme se ujistit, že:
\begin{itemize}
\item se shoduje počet řádek i počet sloupců a že je na každém řádku stejný počet zadaných hodnot
\item se žádná hodnota v zadání neopakuje
\item zadání obsahuje prázdné místo
\item jsou zadány nejméně 3 řádky a sloupce (pro nižší hodnoty je zadání triviální)
\end{itemize}



\subsection{Datové struktury}

\subsection{Algoritmy}

\subsection{Výběr vhodných datových struktur a algoritmů}

\section{Popis implementace}

\section{Uživatelská příručka}

\section{Závěr}

\end{document}